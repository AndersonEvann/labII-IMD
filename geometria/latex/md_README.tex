\section*{turma\+: L\+P1}

\section*{professor\+: Silvio Costa Sampaio}

\section*{Aluno\+: Anderson Evangelista da Silva}

\section*{Matricula\+: 20160115963}

\#compilar\+: make \#rodar o programa\+: deixarei um exemplo de cada sólido geometrico abaixo, lembrando que no programa é possivel ver como fazer para rodar o programa corretamente, basta digitar o nome de um solido geometrico errado ou que não exista. \#\+Exemplos\+: \#./geometria triangulo 2 \#./geometria retangulo 2 4 \#./geometria quadrado 2 \#./geometria circulo 2 \#./geometria piramide 2 4 \#./geometria cubo 2 \#./geometria paralelepipedo 2 4 6 \#./geometria esfera 2 \#\+O\+BS\+: todos os sólidos geometricos são escritos sem acentuação gráfica, e os valores podem ser diferentes de int.

\#dificuldades\+: As minhas maiores dificuldades foram as de conversão de string literal para double e como trabalhar com argv\mbox{[}\mbox{]}. Só tinha utilidado os dois em C, e de formar simples, não conhecia a utilização de ambos em C++ e precisei da ajuda de referencias em inglês para pode utiliza-\/los em meu código. 